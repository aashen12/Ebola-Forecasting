% Options for packages loaded elsewhere
\PassOptionsToPackage{unicode}{hyperref}
\PassOptionsToPackage{hyphens}{url}
%
\documentclass[
]{article}
\usepackage{lmodern}
\usepackage{amssymb,amsmath}
\usepackage{ifxetex,ifluatex}
\ifnum 0\ifxetex 1\fi\ifluatex 1\fi=0 % if pdftex
  \usepackage[T1]{fontenc}
  \usepackage[utf8]{inputenc}
  \usepackage{textcomp} % provide euro and other symbols
\else % if luatex or xetex
  \usepackage{unicode-math}
  \defaultfontfeatures{Scale=MatchLowercase}
  \defaultfontfeatures[\rmfamily]{Ligatures=TeX,Scale=1}
\fi
% Use upquote if available, for straight quotes in verbatim environments
\IfFileExists{upquote.sty}{\usepackage{upquote}}{}
\IfFileExists{microtype.sty}{% use microtype if available
  \usepackage[]{microtype}
  \UseMicrotypeSet[protrusion]{basicmath} % disable protrusion for tt fonts
}{}
\makeatletter
\@ifundefined{KOMAClassName}{% if non-KOMA class
  \IfFileExists{parskip.sty}{%
    \usepackage{parskip}
  }{% else
    \setlength{\parindent}{0pt}
    \setlength{\parskip}{6pt plus 2pt minus 1pt}}
}{% if KOMA class
  \KOMAoptions{parskip=half}}
\makeatother
\usepackage{xcolor}
\IfFileExists{xurl.sty}{\usepackage{xurl}}{} % add URL line breaks if available
\IfFileExists{bookmark.sty}{\usepackage{bookmark}}{\usepackage{hyperref}}
\hypersetup{
  pdftitle={Ebola Forecasting - Residual Analysis},
  pdfauthor={Andy Shen},
  hidelinks,
  pdfcreator={LaTeX via pandoc}}
\urlstyle{same} % disable monospaced font for URLs
\usepackage[margin=1in]{geometry}
\usepackage{longtable,booktabs}
% Correct order of tables after \paragraph or \subparagraph
\usepackage{etoolbox}
\makeatletter
\patchcmd\longtable{\par}{\if@noskipsec\mbox{}\fi\par}{}{}
\makeatother
% Allow footnotes in longtable head/foot
\IfFileExists{footnotehyper.sty}{\usepackage{footnotehyper}}{\usepackage{footnote}}
\makesavenoteenv{longtable}
\usepackage{graphicx,grffile}
\makeatletter
\def\maxwidth{\ifdim\Gin@nat@width>\linewidth\linewidth\else\Gin@nat@width\fi}
\def\maxheight{\ifdim\Gin@nat@height>\textheight\textheight\else\Gin@nat@height\fi}
\makeatother
% Scale images if necessary, so that they will not overflow the page
% margins by default, and it is still possible to overwrite the defaults
% using explicit options in \includegraphics[width, height, ...]{}
\setkeys{Gin}{width=\maxwidth,height=\maxheight,keepaspectratio}
% Set default figure placement to htbp
\makeatletter
\def\fps@figure{htbp}
\makeatother
\setlength{\emergencystretch}{3em} % prevent overfull lines
\providecommand{\tightlist}{%
  \setlength{\itemsep}{0pt}\setlength{\parskip}{0pt}}
\setcounter{secnumdepth}{5}

\title{Ebola Forecasting - Residual Analysis}
\author{Andy Shen}
\date{9/11/2020}

\begin{document}
\maketitle

\hypertarget{setup}{%
\section{Setup}\label{setup}}

We define the beginning, middle, and end of the outbreak such that the
beginning is from May 3, 2018 to September 4, 2018, the middle from
September 5, 2018 to August 25, 2019, and the end from August 26, 2019
to May 4, 2020.

\hypertarget{new-ebola-cases-by-day}{%
\section{New Ebola Cases by Day}\label{new-ebola-cases-by-day}}

The figure below shows the number of new Ebola cases in West Africa by
day during the outbreak. The grey lines represent boundaries between the
beginning, middle and end of the outbreak (see section
\protect\hyperlink{setup}{1}).

\includegraphics{residual_analysis_files/figure-latex/unnamed-chunk-4-1.pdf}

\pagebreak

\hypertarget{hawkes-vs-recursive-7-day-residual-plot}{%
\section{Hawkes vs Recursive 7-Day Residual
Plot}\label{hawkes-vs-recursive-7-day-residual-plot}}

\includegraphics{residual_analysis_files/figure-latex/unnamed-chunk-5-1.pdf}

\pagebreak

\hypertarget{hawkes-vs-recursive-14-day-residual-plot}{%
\section{Hawkes vs Recursive 14-Day Residual
Plot}\label{hawkes-vs-recursive-14-day-residual-plot}}

\begin{verbatim}
## Warning: Removed 2 rows containing missing values (geom_point).
\end{verbatim}

\includegraphics{residual_analysis_files/figure-latex/unnamed-chunk-6-1.pdf}

\pagebreak

\hypertarget{hawkes-vs-recursive-21-day-residual-plot}{%
\section{Hawkes vs Recursive 21-Day Residual
Plot}\label{hawkes-vs-recursive-21-day-residual-plot}}

\begin{verbatim}
## Warning: Removed 2 rows containing missing values (geom_point).
\end{verbatim}

\includegraphics{residual_analysis_files/figure-latex/unnamed-chunk-7-1.pdf}

\pagebreak

\hypertarget{hawkes-only-residual-plot}{%
\section{Hawkes-Only Residual Plot}\label{hawkes-only-residual-plot}}

\includegraphics{residual_analysis_files/figure-latex/unnamed-chunk-8-1.pdf}

\pagebreak

\hypertarget{recursive-only-residual-plot}{%
\section{Recursive-Only Residual
Plot}\label{recursive-only-residual-plot}}

\includegraphics{residual_analysis_files/figure-latex/unnamed-chunk-9-1.pdf}

\pagebreak

\hypertarget{rmse-calculation-during-outbreak}{%
\section{RMSE Calculation During
Outbreak}\label{rmse-calculation-during-outbreak}}

\hypertarget{rmse-calculation}{%
\subsection{RMSE Calculation}\label{rmse-calculation}}

The table below shows the RMSE values from both the Hawkes and Recursive
models during the beginning, middle, and end of the outbreak. The dates
for each section are defined in section \protect\hyperlink{setup}{1}.

\begin{longtable}[]{@{}lrr@{}}
\caption{7-Day Hawkes and Recursive RMSE values during the
outbreak}\tabularnewline
\toprule
& Hawkes & Recursive\tabularnewline
\midrule
\endfirsthead
\toprule
& Hawkes & Recursive\tabularnewline
\midrule
\endhead
Beginning & 5.701 & 5.879\tabularnewline
Middle & 30.740 & 31.768\tabularnewline
End & 3.804 & 4.354\tabularnewline
\bottomrule
\end{longtable}

\begin{longtable}[]{@{}lrr@{}}
\caption{14-Day Hawkes and Recursive RMSE values during the
outbreak}\tabularnewline
\toprule
& Hawkes & Recursive\tabularnewline
\midrule
\endfirsthead
\toprule
& Hawkes & Recursive\tabularnewline
\midrule
\endhead
Beginning & 6.586 & 12.983\tabularnewline
Middle & 63.872 & 64.719\tabularnewline
End & 8.577 & 8.850\tabularnewline
\bottomrule
\end{longtable}

\begin{longtable}[]{@{}lrr@{}}
\caption{21-Day Hawkes and Recursive RMSE values during the
outbreak}\tabularnewline
\toprule
& Hawkes & Recursive\tabularnewline
\midrule
\endfirsthead
\toprule
& Hawkes & Recursive\tabularnewline
\midrule
\endhead
Beginning & 12.200 & 22.551\tabularnewline
Middle & 96.881 & 98.310\tabularnewline
End & 13.462 & 14.545\tabularnewline
\bottomrule
\end{longtable}

It is clear that both models are most accurate in the beginning and end
of the pandemic as opposed to the middle portion.

\end{document}
