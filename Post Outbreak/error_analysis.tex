% Options for packages loaded elsewhere
\PassOptionsToPackage{unicode}{hyperref}
\PassOptionsToPackage{hyphens}{url}
%
\documentclass[
]{article}
\usepackage{lmodern}
\usepackage{amssymb,amsmath}
\usepackage{ifxetex,ifluatex}
\ifnum 0\ifxetex 1\fi\ifluatex 1\fi=0 % if pdftex
  \usepackage[T1]{fontenc}
  \usepackage[utf8]{inputenc}
  \usepackage{textcomp} % provide euro and other symbols
\else % if luatex or xetex
  \usepackage{unicode-math}
  \defaultfontfeatures{Scale=MatchLowercase}
  \defaultfontfeatures[\rmfamily]{Ligatures=TeX,Scale=1}
\fi
% Use upquote if available, for straight quotes in verbatim environments
\IfFileExists{upquote.sty}{\usepackage{upquote}}{}
\IfFileExists{microtype.sty}{% use microtype if available
  \usepackage[]{microtype}
  \UseMicrotypeSet[protrusion]{basicmath} % disable protrusion for tt fonts
}{}
\makeatletter
\@ifundefined{KOMAClassName}{% if non-KOMA class
  \IfFileExists{parskip.sty}{%
    \usepackage{parskip}
  }{% else
    \setlength{\parindent}{0pt}
    \setlength{\parskip}{6pt plus 2pt minus 1pt}}
}{% if KOMA class
  \KOMAoptions{parskip=half}}
\makeatother
\usepackage{xcolor}
\IfFileExists{xurl.sty}{\usepackage{xurl}}{} % add URL line breaks if available
\IfFileExists{bookmark.sty}{\usepackage{bookmark}}{\usepackage{hyperref}}
\hypersetup{
  pdftitle={Ebola Forecasting - Error Analysis},
  pdfauthor={Frederic Schoenberg, Sarita Lee, Andy Shen},
  hidelinks,
  pdfcreator={LaTeX via pandoc}}
\urlstyle{same} % disable monospaced font for URLs
\usepackage[margin=1in]{geometry}
\usepackage{longtable,booktabs}
% Correct order of tables after \paragraph or \subparagraph
\usepackage{etoolbox}
\makeatletter
\patchcmd\longtable{\par}{\if@noskipsec\mbox{}\fi\par}{}{}
\makeatother
% Allow footnotes in longtable head/foot
\IfFileExists{footnotehyper.sty}{\usepackage{footnotehyper}}{\usepackage{footnote}}
\makesavenoteenv{longtable}
\usepackage{graphicx,grffile}
\makeatletter
\def\maxwidth{\ifdim\Gin@nat@width>\linewidth\linewidth\else\Gin@nat@width\fi}
\def\maxheight{\ifdim\Gin@nat@height>\textheight\textheight\else\Gin@nat@height\fi}
\makeatother
% Scale images if necessary, so that they will not overflow the page
% margins by default, and it is still possible to overwrite the defaults
% using explicit options in \includegraphics[width, height, ...]{}
\setkeys{Gin}{width=\maxwidth,height=\maxheight,keepaspectratio}
% Set default figure placement to htbp
\makeatletter
\def\fps@figure{htbp}
\makeatother
\setlength{\emergencystretch}{3em} % prevent overfull lines
\providecommand{\tightlist}{%
  \setlength{\itemsep}{0pt}\setlength{\parskip}{0pt}}
\setcounter{secnumdepth}{5}

\title{Ebola Forecasting - Error Analysis}
\author{Frederic Schoenberg, Sarita Lee, Andy Shen}
\date{}

\begin{document}
\maketitle

\hypertarget{data-input-and-cleaning}{%
\section{Data Input and Cleaning}\label{data-input-and-cleaning}}

We assume the most accurate dataset is the most recent dataset of the
outbreak. We tally the cases such that there is a running total of
infections at each date. This dataset represents the true number of
cases at any given point during the outbreak.

We then import the projections from the Hawkes and Recursive models. For
these predictions, the date preceding the forecasts is the last date of
that dataset with at least one case. The forecasted numbers then predict
the additional number of infections 7, 14, and 21 days after that date,
respectively.

Some dates in the Hawkes forecast models do not have a corresponding
Recursive model forecast, so we omit those values from our analysis.

\hypertarget{hawkes-complete-outbreak-analysis}{%
\section{Hawkes Complete Outbreak
Analysis}\label{hawkes-complete-outbreak-analysis}}

\hypertarget{day-forecast-analysis}{%
\subsection{7-Day Forecast Analysis}\label{day-forecast-analysis}}

Figure 1 below shows the Hawkes 7-Day Forecasts for all recorded
simulations with respect to the recorded number of infections. The RMSE
values for all forecasts are included in Table 1.\newline

\includegraphics{error_analysis_files/figure-latex/unnamed-chunk-4-1.pdf}

The 7-day Hawkes model generally shows accurate predictions throughout
the duration of the pandemic, with slight under-prediction in 2019
during the middle of the pandemic. The Hawkes 7-day forecasts appear to
predict the case counts at the beginning and end of the pandemic quite
accurately.

\pagebreak

\hypertarget{day-forecast-analysis-1}{%
\subsection{14-Day Forecast Analysis}\label{day-forecast-analysis-1}}

Figure 2 below shows the Hawkes 14-Day Forecasts for all recorded
simulations with respect to the recorded number of infections.\newline

\includegraphics{error_analysis_files/figure-latex/unnamed-chunk-5-1.pdf}

The 14-day Hawkes predictions tend to slightly under-predict the true
case counts in the beginning and middle portions of the pandemic, with
more accurate prediction in 2020 as the pandemic comes to an end. The
largest prediction discrepancy is during 2019.

\pagebreak

\hypertarget{day-forecast-analysis-2}{%
\subsection{21-Day Forecast Analysis}\label{day-forecast-analysis-2}}

Figure 3 below shows the Hawkes 21-Day Forecasts for all recorded
simulations with respect to the recorded number of infections.\newline

\includegraphics{error_analysis_files/figure-latex/unnamed-chunk-6-1.pdf}

The 21-day Hawkes projections follow a similar pattern as 14-day Hawkes
projections, with large under-predictions in the middle of the pandemic
and more accurate projections in 2020 towards the end.

\pagebreak

\hypertarget{recursive-complete-outbreak-analysis}{%
\section{Recursive Complete Outbreak
Analysis}\label{recursive-complete-outbreak-analysis}}

\hypertarget{day-forecast-analysis-3}{%
\subsection{7-Day Forecast Analysis}\label{day-forecast-analysis-3}}

Figure 4 below shows the Recursive 7-Day Forecasts for all recorded
simulations with respect to the recorded number of infections.\newline

\includegraphics{error_analysis_files/figure-latex/unnamed-chunk-7-1.pdf}

In general, the 7-day Recursive projections tend to under-predict the
actual case counts. The largest errors generally occur during mid-2019,
which is in the middle of the pandemic, whereas the model tends to have
better prediction in the beginning and towards the end of the pandemic.

\pagebreak

\hypertarget{day-forecast-analysis-4}{%
\subsection{14-Day Forecast Analysis}\label{day-forecast-analysis-4}}

Figure 5 below shows the Recursive 14-Day Forecasts for all simulations
with respect to the recorded number of infections.\newline

\includegraphics{error_analysis_files/figure-latex/unnamed-chunk-8-1.pdf}

The 14-day Recursive forecasts tend to under-predict the actual case
counts during most of the pandemic, and has better prediction at the end
of the pandemic.

\pagebreak

\hypertarget{day-forecast-analysis-5}{%
\subsection{21-Day Forecast Analysis}\label{day-forecast-analysis-5}}

Figure 6 below shows the Recursive 21-Day Forecasts for all recorded
simulations with respect to the recorded number of infections.\newline

\includegraphics{error_analysis_files/figure-latex/unnamed-chunk-9-1.pdf}

Similar to the 14-day forecasts, the 21-day Recursive forecasts tend to
under-predict the true case counts in the beginning and middle of the
pandemic, but is more accurate towards the end of the pandemic with
slight over-prediction.

\pagebreak

\hypertarget{rmse-for-full-hawkes-and-recursive-datasets}{%
\section{RMSE for Full Hawkes and Recursive
Datasets}\label{rmse-for-full-hawkes-and-recursive-datasets}}

We compute the Root Mean Square Error (RMSE) of the 7, 14, and 21-day
forecasts for both the Hawkes and Recursive models. The RMSE is computed
as

\[
\begin{aligned}
RMSE = \sqrt{\frac{\sum_{i=1}^{N}(y_i - \hat{y}_i)^2}{N}}
\end{aligned}
\]

where \(N\) is the total number of observations.

The table below (Table 1) shows the RMSE values for the Hawkes and
Recursive models, with respect to every simulated forecast during the
outbreak.

\begin{longtable}[]{@{}lrr@{}}
\caption{RMSE values for Hawkes and Recursive Models for all
datasets.}\tabularnewline
\toprule
& Hawkes & Recursive\tabularnewline
\midrule
\endfirsthead
\toprule
& Hawkes & Recursive\tabularnewline
\midrule
\endhead
7-day & 28.52 & 29.99\tabularnewline
14-day & 59.45 & 60.98\tabularnewline
21-day & 89.80 & 92.66\tabularnewline
\bottomrule
\end{longtable}

We see that the Hawkes model forecasts have a consistently lower RMSE
than those of the Recursive model for all three prediction days, when
looking at all simulations.

\pagebreak

\hypertarget{omission-of-repeated-entries}{%
\section{Omission of Repeated
Entries}\label{omission-of-repeated-entries}}

Many of the forecasts were run with the same date. These extra runs are
likely due to minor adjustments in the previously recorded data, so we
refine our data to omit any repeated forecasts and only consider the
most recent forecast with a repeated date. Therefore, for multiple
forecasts that ended on the same date, we select the entry furthest down
in the dataset, as it denotes the set with the most recent numbers.

The trend for this analysis very closely mirrors that of the previous
analysis in terms of prediction, as evidenced in figures 7-12 below.

\hypertarget{hawkes-analysis}{%
\subsection{Hawkes Analysis}\label{hawkes-analysis}}

\includegraphics{error_analysis_files/figure-latex/unnamed-chunk-12-1.pdf}

\includegraphics{error_analysis_files/figure-latex/unnamed-chunk-13-1.pdf}

\includegraphics{error_analysis_files/figure-latex/unnamed-chunk-14-1.pdf}

\hypertarget{recursive-analysis}{%
\subsection{Recursive Analysis}\label{recursive-analysis}}

\includegraphics{error_analysis_files/figure-latex/unnamed-chunk-15-1.pdf}

\includegraphics{error_analysis_files/figure-latex/unnamed-chunk-16-1.pdf}

\includegraphics{error_analysis_files/figure-latex/unnamed-chunk-17-1.pdf}

\begin{longtable}[]{@{}lrr@{}}
\caption{RMSE values for Hawkes and Recursive Models for refined
dataests.}\tabularnewline
\toprule
& Hawkes & Recursive\tabularnewline
\midrule
\endfirsthead
\toprule
& Hawkes & Recursive\tabularnewline
\midrule
\endhead
7-day & 26.98 & 29.92\tabularnewline
14-day & 56.83 & 61.65\tabularnewline
21-day & 86.24 & 91.98\tabularnewline
\bottomrule
\end{longtable}

There is not a large difference in RMSE of the full forecast analysis
compared with that from the refined forecasts with the repeated dates
removed.

\end{document}
